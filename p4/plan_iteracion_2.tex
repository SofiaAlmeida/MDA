\documentclass[11pt]{article}
\usepackage{fontspec}
\usepackage[spanish]{babel}
\usepackage{listings}
\usepackage{graphicx}
\graphicspath{{../Imagenes/}}

\usepackage[paper=portrait, pagesize]{typearea}
\usepackage{titlepic}

%%% Tablas
\usepackage{longtable}
\usepackage{tabularx}
\usepackage{float}
\usepackage{adjustbox}
\usepackage{booktabs}
\usepackage{multirow}
\usepackage[dvipsnames]{xcolor,colortbl}
\definecolor{LightCyan}{rgb}{0.88,1,1}
\definecolor{dollarbill}{rgb}{0.74, 0.92, 0.6}
\renewcommand{\arraystretch}{1.7}

%%%%%%%%%% Contador HU
\newcounter{HUCounter}
\newcommand{\hu}[1]{\refstepcounter{HUCounter}\textbf{\rmfamily HU-\theHUCounter}\label{#1}}

%%% Referencias cruzadas HU
\usepackage{xr}
\externaldocument{../p3/HistoriasUsuario}
\externaldocument{./plan_iteracion_1}

\begin{document}

\begin{titlepage}
\centering
\vspace{4.5cm}
{\scshape\LARGE Plan de la iteración 2\par}
\vspace{1.5cm}

\includegraphics[width=10cm] {Logo}

\vspace{3cm}
{\scshape\large \par}
\vspace{1cm}

{Miguel Albertí Pons\\
Sofía Almeida Bruno\\
Pedro Manuel Flores Crespo\\
María Victoria Granados Pozo\\
Lidia Martín Chica
\par}

\end{titlepage}

\newpage

\section{Actualización Product Backlog}
Tras añadir las historias de usuario surgidas en la revisión del sprint anterior, la pila del producto queda de la siguiente manera:

\begin{longtable}{p{0.12\linewidth}p{0.7\linewidth}p{0.07\linewidth}p{0.07\linewidth}}
    \toprule
    \textbf{Id HU} & \textbf{Título} & \textbf{Est.} & \textbf{Prio.}\\
    \midrule
    \rowcolor{dollarbill}
    \hu{hu:registro} & Un usuario quiere registrarse en la aplicación para poder planificar su viaje & 2 & 1\\
    \rowcolor{dollarbill}
    \hu{hu:identificarse} & Un usuario quiere identificarse en la aplicación para  acceder a su contenido & 1 & 1\\
    \rowcolor{dollarbill}
    \hu{hu:ver_publicaciones} & Un usuario quiere ver las publicaciones para saber cuáles le pueden interesar & 2 & 2\\
    \rowcolor{dollarbill}
    \hu{hu:subir} & Un usuario quiere subir una publicación para compartir su experiencia & 3 & 2\\
    \hu{hu:valorar} & Un usuario quiere valorar un viaje que ha realizado para mostrar su opinión & 2 & 4\\
    \hu{hu:comentar} & Un usuario quiere comentar en una publicación para preguntar una duda o aportar información & 2 & 4\\ 
    \hu{hu:ver_coment} & Un usuario quiere ver los comentarios de una publicación para conocer las experiencias y opiniones de otros usuarios & 1 & 4\\ 
    \hu{hu:ver_val} & Un usuario quiere ver las valoraciones de una publicación para tenerlo en cuenta a la hora de tomar una decisión & 1 & 4\\ 
    \hu{hu:filtrar} & Un usuario quiere filtrar la búsqueda de las publicaciones para encontrar una que se adapte a sus intereses & - & -\\ 
    \rowcolor{dollarbill}
    \textbf{HU-\ref{hu:filtrar}.1} & Un usuario quiere filtrar la búsqueda de las publicaciones por lugar & 1/2 & 3\\ 
    \rowcolor{dollarbill}
    \textbf{HU-\ref{hu:filtrar}.2} & Un usuario quiere filtrar la búsqueda de las publicaciones por fecha & 1/2 & 3\\ 
    \rowcolor{dollarbill}
    \textbf{HU-\ref{hu:filtrar}.3} & Un usuario quiere filtrar la búsqueda de las publicaciones por tipo de publicación (paquete, ruta, actividad, evento) & 1/2 & 3\\ 
    \rowcolor{dollarbill}
    \textbf{HU-\ref{hu:filtrar}.4} & Un usuario quiere filtrar la búsqueda de las publicaciones por tipo de actividad (cultural, deportiva, etc)) & 1/2 & 3\\ 
    \textbf{HU-\ref{hu:filtrar}.5} & Un usuario quiere filtrar la búsqueda de las publicaciones para que aparezcan las más visitadas & 1/2 & 4\\ 
    \textbf{HU-\ref{hu:filtrar}.6} & Un usuario quiere filtrar la búsqueda de las publicaciones para que aparezcan las más realizadas & 1/2 & 4\\
    \hu{hu:personalizar} & Un usuario quiere personalizar un paquete para adaptarlo a sus gustos & - & - \\ 
    \rowcolor{dollarbill}
    \textbf{HU-\ref{hu:personalizar}.1} & Un usuario quiere eliminar una actividad de un paquete & 1/2 & 3 \\ 
    \textbf{HU-\ref{hu:personalizar}.2} & Un usuario quiere añadir una actividad de un paquete & 1 & 3 \\ 
    \textbf{HU-\ref{hu:personalizar}.3} & Un usuario quiere modificar el orden de las actividades de un paquete & 3/2 & 4 \\
    \rowcolor{dollarbill}
    \hu{hu:seleccionar} & Un usuario quiere seleccionar un viaje/actividad para realizarlo & 3/2 & 2\\ 
    \hu{hu:guardar} & Un usuario quiere guardar un viaje/actividad para verlo en otro momento & 1 & 5\\ 
    \hu{hu:historial} & Un usuario quiere ver su historial de viajes guardados/lista de deseos para elegir entre ellos & 3/2 & 5\\
    \hu{hu:maleta} & Un usuario quiere gestionar su maleta & - & -\\
    \textbf{HU-\ref{hu:maleta}.1} & Un usuario quiere obtener una recomendación del vestuario que debe llevar según el pronóstico meteorológico & 5 & 6\\
    \textbf{HU-\ref{hu:maleta}.2} & Un usuario quiere obtener una recomendación del vestuario que debe llevar según el tipo de actividad que va a realizar & 4 & 6\\
    \textbf{HU-\ref{hu:maleta}.3} & Un usuario quiere obtener una recomendación del vestuario que debe llevar según la duración del viaje & 4 & 6\\
    \hu{hu:gaudio} & Un usuario quiere grabar un audioguía sobre su ciudad para compartirla con otros usuarios & 2 & 6\\
    \hu{hu:eaudio} & Un usuario quiere escuchar un audioguía sobre la ruta que va a realizar para informarse sobre la ciudad & 3/2 & 6\\
%TODO: AÑADIR NUEVAS
	\hu{hu:buscar} & Un usuario quiere buscar una publicación a partir de un texto & - & - \\
	\hu{hu:ver_datos} & Un usuario quiere ver su información personal & - & - \\
	\hu{hu:modificar_datos} & Un usuario quiere modificar sus datos personales & - & - \\
	\hu{hu:marcar_realizado} & Un usuario quiere indicar que ha realizado una actividad & - & - \\
	\hu{hu:cerrar_sesion} & Un usuario desea cerrar la sesión en la aplicación & - & - \\
	\hu{hu:ver_publicadas} & Un usuario quiere ver sus propias publicaciones & - & - \\
	\hu{hu:eliminar_deseo} & Un usuario quiere eliminar una actividad de su lista de deseos & - & - \\
	\hu{hu:eliminar_voy_a_realizar} & Un usuario desea eliminar una actividad de su lista de actividades a realizar & - & - \\
	\hu{hu:recuperar_contraseña} & Un usuario ha olvidado su contraseña y desea poder recuperarla & - & - \\
	\hu{hu:saber_mas} & Un usuario desea conocer más información sobre una publicación en particular & - & - \\
	\hu{hu:sobre_nosotros} & Un usuario quiere saber más informacón sobre quién ha desarrollado la aplicación & - & - \\
	\hu{hu:ayuda} & Un usuario desea ver una guía sobre el uso de la aplicación & - & - \\
	\hu{eliminar_publicacion} & Un usuario desea eliminar una publicación que ha subido & - & - \\
    \bottomrule
\end{longtable}


\section{Objetivos de la iteración}
El objetivo que se ha diseñado para esta primera iteración es el prototipado de las funcionalidades básicas, como registro e identificación, así como subir una publicación y seleccionar un viaje.

Esta iteración generará prototipos visuales para que el cliente pueda ver cómo serán las funcionalidades básicas de la aplicación.

\section{Listado inicial de HU a desarrollar}
\begin{longtable}{p{0.13\linewidth}p{0.65\linewidth}p{0.05\linewidth}p{0.05\linewidth}p{0.05\linewidth}}
	\toprule
	\textbf{Id HU} & \textbf{Título} & \textbf{Est.}\\
	\midrule
	\textbf{HU-\ref{hu:registro}} & Un usuario quiere registrarse en la aplicación para poder planificar su viaje & 2\\
	\textbf{HU-\ref{hu:identificarse}} & Un usuario quiere identificarse en la aplicación para  acceder a su contenido & 1\\
	\textbf{HU-\ref{hu:ver_publicaciones}} & Un usuario quiere ver las publicaciones para saber cuáles le pueden interesar & 2\\
	\textbf{HU-\ref{hu:subir}} & Un usuario quiere subir una publicación para compartir su experiencia & 3\\
	\textbf{HU-\ref{hu:filtrar}.1} & Un usuario quiere filtrar la búsqueda de las publicaciones por lugar & 1/2 \\ 
	\textbf{HU-\ref{hu:filtrar}.2} & Un usuario quiere filtrar la búsqueda de las publicaciones por fecha & 1/2 \\
	\textbf{HU-\ref{hu:filtrar}.3} & Un usuario quiere filtrar la búsqueda de las publicaciones por tipo de publicación (paquete, ruta, actividad, evento) & 1/2 \\
	\textbf{HU-\ref{hu:filtrar}.4} & Un usuario quiere filtrar la búsqueda de las publicaciones por tipo de actividad (cultural, deportiva, etc)) & 1/2 \\
	\textbf{HU-\ref{hu:personalizar}.1} & Un usuario quiere eliminar una actividad de un paquete & 1/2\\ 
	\textbf{HU-\ref{hu:seleccionar}} & Un usuario quiere seleccionar un viaje/actividad para realizarlo & 3/2 \\
	\bottomrule
\end{longtable}

\section{Descomposición en tareas de desarrollo}
Se incluye el resultado de la descomposición de las historias de usuario en tareas de desarrollo, así como la asignación a los desarrolladores y la estimación realizada de su duración (en horas).

%% Registro
%\centering
\begin{longtable}{p{0.18\linewidth}|p{0.6\linewidth}|p{0.2\linewidth}}
  \rowcolor{LightCyan}
  \textbf{Identificador} & \textbf{HU-\ref{hu:registro}}. Un usuario quiere registrarse en la aplicación para poder planificar su viaje & \textbf{Est. (PH):} 2 \\  
  \bottomrule
\end{longtable}
\vspace{-0.5cm}
\begin{longtable}{p{0.18\linewidth}|p{0.4\linewidth}|p{0.18\linewidth}|p{0.2\linewidth}}
  \toprule
  \textbf{Identificador} & \textbf{Título de la tarea de desarrollo} & \textbf{Est. (horas)} & \textbf{Desarrollador} \\
  Tarea \ref{hu:registro} - 1 & Crear la base de datos con los usuarios & 2 & MVGP\\
  Tarea \ref{hu:registro} - 2 & Crear comando para insertar el usuario en la base de datos & 1 & SAB\\
  Tarea \ref{hu:registro} - 3 & Crear la interfaz de usuario & 1 & LMC\\
  Tarea \ref{hu:registro} - 4 & Realizar pruebas de aceptación & 4 & PFC \\
  Tarea \ref{hu:registro} - 5 & Crear documentación & 2 & MAP\\
  \bottomrule
\end{longtable}
\vspace{-0.5cm}
\begin{longtable}{p{1.028\linewidth}}
  \textbf{Observaciones}\\
  \midrule
  Ninguna\\
  \bottomrule
\end{longtable}

%% Identificación
%\centering
\begin{longtable}{p{0.18\linewidth}|p{0.6\linewidth}|p{0.2\linewidth}}
  \rowcolor{LightCyan}
  \textbf{Identificador} & \textbf{HU-\ref{hu:identificarse}}. Un usuario quiere identificarse en la aplicación para acceder a su contenido & \textbf{Est. (PH):} 2 \\
  \bottomrule
\end{longtable}
\vspace{-0.5cm}
\begin{longtable}{p{0.18\linewidth}|p{0.4\linewidth}|p{0.18\linewidth}|p{0.2\linewidth}}
  \toprule
  \textbf{Identificador} & \textbf{Título de la tarea de desarrollo} & \textbf{Est. (horas)} & \textbf{Desarrollador} \\
  Tarea \ref{hu:identificarse}.2 - 1 & Crear interfaz de usuario & 1 & LMC\\
  Tarea \ref{hu:identificarse}.2 - 2 & Realizar pruebas de aceptación & 4 & MAP\\
  Tarea \ref{hu:identificarse}.2 - 3 & Crear documentación & 2 & LMC\\
  \bottomrule
\end{longtable}
\vspace{-0.5cm}
\begin{longtable}{p{1.028\linewidth}}
  \textbf{Observaciones}\\
  \midrule
  Ninguna\\
  \bottomrule
\end{longtable}

%% Ver publicaciones
%\centering
\begin{longtable}{p{0.18\linewidth}|p{0.6\linewidth}|p{0.2\linewidth}}
  \rowcolor{LightCyan}
  \textbf{Identificador} & \textbf{HU-\ref{hu:ver_publicaciones}}. Un usuario quiere ver las publicaciones para saber cuáles le pueden interesar & \textbf{Est. (PH):} 2\\	
  \bottomrule
\end{longtable}
\vspace{-0.5cm}
\begin{longtable}{p{0.18\linewidth}|p{0.5\linewidth}|p{0.1\linewidth}|p{0.2\linewidth}}
  \toprule
  \textbf{Identificador} & \textbf{Título de la tarea de desarrollo} & \textbf{Est.} & \textbf{Desarrollador} \\
  Tarea \ref{hu:ver_publicaciones} - 1 & Acceder  a la base de datos y mostrar las actividades que se encuentran cerca de él & 2 & PFC\\
  Tarea \ref{hu:ver_publicaciones} - 2 & Realizar la interfaz de usuario & 4 & MAP\\
  Tarea \ref{hu:ver_publicaciones} - 3 & Realizar pruebas de aceptación & 1 & LMC\\
  \bottomrule
\end{longtable}
\vspace{-0.5cm}
\begin{longtable}{p{1.028\linewidth}}
  \textbf{Observaciones}\\
  \midrule
  Ninguna\\
  \bottomrule
\end{longtable}

%% Subir
%\centering
\begin{longtable}{p{0.18\linewidth}|p{0.6\linewidth}|p{0.2\linewidth}}
  \rowcolor{LightCyan}
  \textbf{Identificador} & \textbf{HU-\ref{hu:subir}}. Un usuario quiere subir una publicación para compartir su experiencia& \textbf{Est. (PH):} 3 \\
  \bottomrule
\end{longtable}
\vspace{-0.5cm}
\begin{longtable}{p{0.18\linewidth}|p{0.4\linewidth}|p{0.18\linewidth}|p{0.2\linewidth}}
  \toprule
  \textbf{Identificador} & \textbf{Título de la tarea de desarrollo} & \textbf{Est. (horas)} & \textbf{Desarrollador} \\
  Tarea \ref{hu:subir} - 1 & Crear base de datos con las publicaciones & 1 & LMC\\
  Tarea \ref{hu:subir} - 2 & Crear comando para insertar publicación en la base de datos. & 1/2 & SAB \\
  Tarea \ref{hu:subir} - 3 & Crear interfaz de usuario & 4 & PFC\\
  Tarea \ref{hu:subir} - 4 & Realizar pruebas de aceptación & 3 & LMC\\
  Tarea \ref{hu:subir} - 5 & Crear documentación. & 2 & PFC\\
  \bottomrule
\end{longtable}
\vspace{-0.5cm}
\begin{longtable}{p{1.028\linewidth}}
  \textbf{Observaciones}\\
  \midrule
  Ninguna\\
  \bottomrule
\end{longtable}

%% Filtrar por lugar
%\centering
\begin{longtable}{p{0.18\linewidth}|p{0.6\linewidth}|p{0.2\linewidth}}
  \rowcolor{LightCyan}
  \textbf{Identificador} & \textbf{HU-\ref{hu:filtrar}.1}.  Un usuario quiere filtrar la búsqueda de las publicaciones por lugar & \textbf{Est. (PH):} 0.5 \\
  \bottomrule
\end{longtable}
\vspace{-0.5cm}
\begin{longtable}{p{0.18\linewidth}|p{0.4\linewidth}|p{0.18\linewidth}|p{0.2\linewidth}}
  \toprule
  \textbf{Identificador} & \textbf{Título de la tarea de desarrollo} & \textbf{Est. (horas)} & \textbf{Desarrollador} \\
  Tarea \ref{hu:filtrar}.1 - 1 & Realizar diseño de la interfaz que permita incluir o excluir lugares & 5/2 & SAB\\
  Tarea \ref{hu:filtrar}.1 - 2 & Consultar base de datos & 1/2 & MVGP\\
  Tarea \ref{hu:filtrar}.1 - 3 & Realizar pruebas de aceptación & 1 & SAB\\
  Tarea \ref{hu:filtrar}.1 - 4 & Crear documentación & 1 & MVGP\\
  Tarea \ref{hu:filtrar}.1 - 5 & Integrar con el resto con el resto de filtros & 1/2 & MAP\\
  \bottomrule
\end{longtable}
\vspace{-0.5cm}
\begin{longtable}{p{1.028\linewidth}}
  \textbf{Observaciones}\\
  \midrule
  Ninguna\\
  \bottomrule
\end{longtable}

%% Filtrar por fecha
%\centering
%\begin{adjustbox}{width=1.2\textwidth}
\begin{longtable}{p{0.18\linewidth}|p{0.6\linewidth}|p{0.2\linewidth}}
  \rowcolor{LightCyan}
  \textbf{Identificador} & \textbf{HU-\ref{hu:filtrar}.2}. Un usuario quiere filtrar la búsqueda de las publicaciones por fecha & \textbf{Est. (PH):} 1/2 \\
  \bottomrule
\end{longtable}
\vspace{-0.5cm}
\begin{longtable}{p{0.18\linewidth}|p{0.4\linewidth}|p{0.18\linewidth}|p{0.2\linewidth}}
  \toprule
  \textbf{Identificador} & \textbf{Título de la tarea de desarrollo} & \textbf{Est. (horas)} & \textbf{Desarrollador} \\
  Tarea \ref{hu:filtrar}.2 - 1 & Acceder a la base de datos y obtener las actividades que se encuentran en las fechas indicadas & 1 & SAB\\
  Tarea \ref{hu:filtrar}.2 - 2 & Añadir a la interfaz un calendario para poder filtrar por fecha & 2 & MAP\\
  Tarea \ref{hu:filtrar}.2 - 3 & Realizar pruebas de aceptación & 2 & MVGP\\
  Tarea \ref{hu:filtrar}.2 - 4 & Integrar con el resto de filtros & 1/2 & MVGP\\
  \bottomrule
\end{longtable}
\vspace{-0.5cm}
\begin{longtable}{p{1.028\linewidth}}
  \textbf{Observaciones}\\
  \midrule
  Ninguna\\
  \bottomrule
\end{longtable}
%\end{adjustbox}

%% Filtrar por tipo de publicación
%\centering
\begin{longtable}{p{0.18\linewidth}|p{0.6\linewidth}|p{0.2\linewidth}}
  \rowcolor{LightCyan}
  \textbf{Identificador} & \textbf{HU-\ref{hu:filtrar}.3}. Un usuario quiere filtrar la búsqueda de las publicaciones por tipo de publicación (paquete, ruta, actividad, evento) & \textbf{Est. (PH):} 1/2 \\
  \bottomrule
\end{longtable}
\vspace{-0.5cm}
\begin{longtable}{p{0.18\linewidth}|p{0.4\linewidth}|p{0.18\linewidth}|p{0.2\linewidth}}
  \toprule
  \textbf{Identificador} & \textbf{Título de la tarea de desarrollo} & \textbf{Est. (horas)} & \textbf{Desarrollador} \\
  Tarea \ref{hu:filtrar}.3 - 1 & Diseño de la interfaz de búsqueda seleccionando el tipo de publicación & 2 & MVGP\\
  Tarea \ref{hu:filtrar}.3 - 2 & Realizar las pruebas de aceptación & 1 & SAB\\
  Tarea \ref{hu:filtrar}.3 - 3 & Realizar la búsqueda en la base de datos & 1 & SAB \\
  Tarea \ref{hu:filtrar}.3 - 4 & Integrar con el resto de filtros & 1/2 & LMC\\
  \bottomrule
\end{longtable}


%% Filtrar tipo de actividad
%\centering
\begin{longtable}{p{0.18\linewidth}|p{0.6\linewidth}|p{0.2\linewidth}}
  \rowcolor{LightCyan}
  \textbf{Identificador} & \textbf{HU-\ref{hu:filtrar}.4}. Un usuario quiere filtrar la búsqueda de las publicaciones por tipo de actividad (cultural, deportiva,etc) & \textbf{Est. (PH):} 0.5 \\
  \bottomrule
\end{longtable}
\vspace{-0.5cm}
\begin{longtable}{p{0.18\linewidth}|p{0.4\linewidth}|p{0.18\linewidth}|p{0.2\linewidth}}
  \toprule
  \textbf{Identificador} & \textbf{Título de la tarea de desarrollo} & \textbf{Est. (horas)} & \textbf{Desarrollador} \\
  Tarea \ref{hu:filtrar}.4 - 1 & Realizar diseño de la interfaz que permita incluir o excluir lugares & 2 & SAB\\
  Tarea \ref{hu:filtrar}.4 - 2 & Implementar la búsqueda en la base de datos & 1/2 & SAB \\
  Tarea \ref{hu:filtrar}.4 - 3 & Realizar pruebas de aceptación & 2 & MVGP\\
  Tarea \ref{hu:filtrar}.4 - 4 & Crear documentación & 2 & LMC\\
  Tarea \ref{hu:filtrar}.4 - 5 & Integrar con el resto de filtros & 1/2 & SAB\\
  \bottomrule
\end{longtable}
\vspace{-0.5cm}
\begin{longtable}{p{1.028\linewidth}}
  \textbf{Observaciones}\\
  \midrule
  Ninguna\\
  \bottomrule
\end{longtable}


%% Eliminar actividad paquete o ruta
%\centering
\begin{longtable}{p{0.18\linewidth}|p{0.6\linewidth}|p{0.2\linewidth}}
  \rowcolor{LightCyan}
  \textbf{Identificador} & \textbf{HU-\ref{hu:personalizar}.1}.Un usuario quiere eliminar una actividad de un paquete/ruta & \textbf{Est. (PH):} 1/2 \\
  \bottomrule
\end{longtable}
\vspace{-0.5cm}
\begin{longtable}{p{0.18\linewidth}|p{0.4\linewidth}|p{0.18\linewidth}|p{0.2\linewidth}}
  \toprule
  \textbf{Identificador} & \textbf{Título de la tarea de desarrollo} & \textbf{Est. (horas)} & \textbf{Desarrollador} \\
  Tarea \ref{hu:personalizar}.1 - 1 & Diseño de la interfaz, añadiendo una X al lado de cada actividad & 1 & MVGP\\
  Tarea \ref{hu:personalizar}.1 - 2 & Modificar la ruta en la base de datos & 2 & SAB\\
  Tarea \ref{hu:personalizar}.1 - 3 & Pruebas de aceptación & 1/2 & LMC\\
  \bottomrule
\end{longtable}
\vspace{-0.5cm}
\begin{longtable}{p{1.028\linewidth}}
  \textbf{Observaciones}\\
  \midrule
  Ninguna\\
  \bottomrule
\end{longtable}


%% Seleccionar
%\centering
\begin{longtable}{p{0.18\linewidth}|p{0.6\linewidth}|p{0.2\linewidth}}
  \rowcolor{LightCyan}
  \textbf{Identificador} & \textbf{HU-\ref{hu:seleccionar}}. Un usuario quiere seleccionar un viaje/actividad para realizarlo& \textbf{Est. (PH):} 3/2 \\
  \bottomrule
\end{longtable}
\vspace{-0.5cm}
\begin{longtable}{p{0.18\linewidth}|p{0.4\linewidth}|p{0.18\linewidth}|p{0.2\linewidth}}
  \toprule
  \textbf{Identificador} & \textbf{Título de la tarea de desarrollo} & \textbf{Est. (horas)} & \textbf{Desarrollador} \\
  Tarea \ref{hu:seleccionar} - 1 &  Crear base de datos & 1 & MVGP\\
  Tarea \ref{hu:seleccionar} - 2 & Crear comando para insertar actividad en la base de datos. & 1/2 & MVGP\\
  Tarea \ref{hu:seleccionar} - 3 & Realizar pruebas de aceptación & 1/5 & LMC\\
  Tarea \ref{hu:seleccionar} - 4 & Crear documentación & 1/2 & MVGP\\
  Tarea \ref{hu:seleccionar} - 5 & Integrar selección con la visualización de las publicaciones & 1/2 & PFC\\
  \bottomrule
\end{longtable}
\vspace{-0.5cm}
\begin{longtable}{p{1.028\linewidth}}
  \textbf{Observaciones}\\
  \midrule
  Ninguna\\
  \bottomrule
\end{longtable}


\section{Carga prevista en los desarrolladores}

Información final sobre la carga prevista de trabajo de cada uno de los miembros del equipo de desarrollo en las tareas asignadas en la iteración.

\begin{table}[H]
  \centering

  \begin{adjustbox}{width=\textwidth}
\begin{tabular}{lrrrrr}
  \toprule
  \textbf{Desarrollador} & \textbf{Velocidad inicial} & \textbf{Dedicación} & \textbf{Carga de trabajo} & \textbf{Tareas aceptadas} \\
  & \textbf{(días ideales)} & \textbf{(\% de tiempo)} & \textbf{(días ideales)} & \textbf{ (cantidad)}\\
  \midrule
  MAP & 12 & 60 & 12.5 & 5\\
  SAB & 12 & 60 & 13 & 13\\
  PFC & 12 & 60 & 12.5 & 5\\
  MVGP & 12 & 60 & 13 & 11\\
  LMC & 12 & 60 & 12.2 & 10\\
  \bottomrule
\end{tabular}
\end{adjustbox}
\end{table}

\section{Planificación temporal de la iteración}
La planificación temporal que se ha utilizado para hacer las estimaciones es la siguiente:


\begin{longtable}{lrrrrr}
  \caption*{Semana 1}\\
  \toprule
  \textbf{Desarrollador} & \textbf{Día 1} & \textbf{Día 2} & \textbf{Día 3} & \textbf{Día 4} \\
  \midrule
  MAP &  &  &  &  \\
  SAB &  &  &  & \\
  PFC \\
  MVGP & & \\
  LMC \\
  \bottomrule
\end{longtable}

\begin{longtable}{lrrrrr}
  \caption*{Semana 2}\\
  \toprule
  \textbf{Desarrollador} & \textbf{Día 1} & \textbf{Día 2} & \textbf{Día 3} & \textbf{Día 4} \\
  \midrule
  MAP &  & \\
  SAB &  & \\
  PMFC \\
  MVGP & & \\
  LMC \\
  \bottomrule
\end{longtable}

\section{Desviaciones previstas}

Añadimos las siguientes historias de usuario a la lista del producto:

\end{document}
