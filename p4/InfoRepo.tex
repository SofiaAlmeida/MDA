\documentclass[11pt]{article}
\usepackage{fontspec}
\usepackage[spanish]{babel}
%\usepackage[utf8]{inputenc}
\usepackage{listings}
\usepackage{graphicx}
\graphicspath{{../Imagenes/}}

\usepackage{titlepic}
\usepackage{hyperref}

%%% Tablas
\usepackage{longtable}
\newcommand{\tabitem}{~~\llap{\textbullet}~~}
%% \usepackage{longtablex}
\usepackage{float}
\usepackage{adjustbox}
\usepackage{booktabs}
\usepackage{multirow}
\renewcommand{\arraystretch}{1.7}

%%%%%%%%%% Contador HU
\newcounter{HUCounter}
\newcommand{\hu}[1]{\refstepcounter{HUCounter}\textbf{\rmfamily HU-\theHUCounter}\label{#1}}

\begin{document}

\begin{titlepage}
\centering
\vspace{4.5cm}
{\scshape\LARGE Información para acceder al repositorio \par}
\vspace{1.5cm}

\includegraphics[width=10cm]{Logo}

\vspace{3cm}
{\scshape\large \par}
\vspace{1cm}

{Miguel Albertí Pons\\
Sofía Almeida Bruno\\
Pedro Manuel Flores Crespo\\
María Victoria Granados Pozo\\
Lidia Martín Chica
\par}

\end{titlepage}
\newpage

\section{Enlaces de Trello}
Tal y como indica la práctica, hemos usado la aplicación Trello para gestionar las historias de usuario. 

\begin{itemize}
\item \url{https://trello.com/invite/b/5Wi2rqx5/4f7ea1658934d19280be9b60fb97ffc7/sprint1} \\

Este enlace dirige a un tablero con las historias de usuario incluidas en la primera iteración del proyecto. 

\item \url{https://trello.com/invite/b/8hX1URIX/c4a4d7b67b63f6516e5b531fd5ae79c6/sprint2} \\

Este enlace dirige a un tablero con las historias de usuario incluidas en la segunda iteración del proyecto. 

\item \url{https:....} \\
Este es un enlace al repositorio de GitHub donde se ha ido llevando el trabajo de las memorias del proyecto


\end{itemize}
\end{document}
