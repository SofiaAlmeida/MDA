\documentclass[11pt]{article}
\usepackage{fontspec}
\usepackage[spanish]{babel}
%\usepackage[utf8]{inputenc}
\usepackage{listings}
\usepackage{graphicx}
\graphicspath{{../Imagenes/}}

\usepackage{titlepic}
\usepackage{hyperref}

%%% Tablas
\usepackage{longtable}
\newcommand{\tabitem}{~~\llap{\textbullet}~~}
%% \usepackage{longtablex}
\usepackage{float}
\usepackage{adjustbox}
\usepackage{booktabs}
\usepackage{multirow}
\renewcommand{\arraystretch}{1.7}

%%%%%%%%%% Contador HU
\newcounter{HUCounter}
\newcommand{\hu}[1]{\refstepcounter{HUCounter}\textbf{\rmfamily HU-\theHUCounter}\label{#1}}

\begin{document}

\begin{titlepage}
\centering
\vspace{4.5cm}
{\scshape\LARGE Información para acceder a la web \par}
\vspace{1.5cm}

\includegraphics[width=10cm]{Logo}

\vspace{3cm}
{\scshape\large \par}
\vspace{0.5cm}

{Miguel Albertí Pons\\
Sofía Almeida Bruno\\
Pedro Manuel Flores Crespo\\
María Victoria Granados Pozo\\
Lidia Martín Chica
\par}

\end{titlepage}
\newpage

\section{Enlace a la página web}
El siguiente enlace lleva directamente a nuestra página web: 

\url{https://makeatravelmda.wixsite.com/makeatravel} \\

La página web se ha dividido en tres secciones:
\begin{itemize}
\item Inicio. En ella se describen las principales ventajas que tiene el uso de nuestra aplicación.
\item Kanban. Incluye fotografías con los tableros Kanban que se han generado durante las dos iteraciones que se han llevado a cabo para el desarrollo de la aplicación.
\item Aplicación. Incluye el enlace al repositorio con toda la información del proyecto y capturas de pantalla de los prototipos construidos.
\end{itemize} 
\end{document}
