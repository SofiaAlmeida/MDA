
\documentclass[11pt]{article}
\usepackage{fontspec}
\usepackage[spanish]{babel}
\usepackage{listings}
\usepackage{graphicx}
\graphicspath{{../Imagenes/}}

\usepackage[paper=portrait, pagesize]{typearea}
\usepackage{titlepic}

%%% Tablas
\newcommand{\tabitem}{~~\llap{\textbullet}~~}
\usepackage{longtable}
\usepackage{tabularx}
\usepackage{float}
\usepackage{adjustbox}
\usepackage{booktabs}
\usepackage{multirow}
\usepackage[dvipsnames]{xcolor,colortbl}
\definecolor{LightCyan}{rgb}{0.88,1,1}
\definecolor{dollarbill}{rgb}{0.74, 0.92, 0.6}
\renewcommand{\arraystretch}{1.7}


\begin{document}

\begin{titlepage}
\centering
\vspace{4.5cm}
{\scshape\LARGE Análisis del uso de métodos ágiles\par}
\vspace{1.5cm}

\includegraphics[width=10cm] {Logo}

\vspace{3cm}
{\scshape\large \par}
\vspace{1cm}

{Miguel Albertí Pons\\
Sofía Almeida Bruno\\
Pedro Manuel Flores Crespo\\
María Victoria Granados Pozo\\
Lidia Martín Chica
\par}

\end{titlepage}

\newpage
\section{Análisis del uso de métodos ágiles}
Durante el desarrollo de las prácticas hemos propuesto, diseñado y prototipado una parte de nuestra aplicación, MakeATravel. En el proceso hemos utilizado métodos ágiles con el objetivo de conocerlos en más profundidad, no es lo mismo saber cómo son en teoría a ponerlos en práctica. Ahora que terminamos la asignatura, y con ello las prácticas y nuestro estudio de los métodos ágiles (de momento...), nos paramos a reflexionar sobre las ventajas e inconvenientes que han aportado sobre los métodos tradicionales.

\subsection*{Ventajas}


\subsection*{Inconvenientes}
\end{document}
