
\documentclass[11pt]{article}
\usepackage{fontspec}
\usepackage[spanish]{babel}
\usepackage{listings}
\usepackage{graphicx}
\graphicspath{{../Imagenes/}}

\usepackage[paper=portrait, pagesize]{typearea}
\usepackage{titlepic}

%%% Tablas
\newcommand{\tabitem}{~~\llap{\textbullet}~~}
\usepackage{longtable}
\usepackage{tabularx}
\usepackage{float}
\usepackage{adjustbox}
\usepackage{booktabs}
\usepackage{multirow}
\usepackage[dvipsnames]{xcolor,colortbl}
\definecolor{LightCyan}{rgb}{0.88,1,1}
\definecolor{dollarbill}{rgb}{0.74, 0.92, 0.6}
\renewcommand{\arraystretch}{1.7}


\begin{document}

\begin{titlepage}
\centering
\vspace{4.5cm}
{\scshape\LARGE Análisis del uso de métodos ágiles\par}
\vspace{1.5cm}

\includegraphics[width=10cm] {Logo}

\vspace{3cm}
{\scshape\large \par}
\vspace{1cm}

{Miguel Albertí Pons\\
Sofía Almeida Bruno\\
Pedro Manuel Flores Crespo\\
María Victoria Granados Pozo\\
Lidia Martín Chica
\par}

\end{titlepage}

\newpage
\section{Análisis del uso de métodos ágiles}
Durante el desarrollo de las prácticas hemos propuesto, diseñado y prototipado una parte de nuestra aplicación, MakeATravel. En el proceso hemos utilizado métodos ágiles con el objetivo de conocerlos en más profundidad, no es lo mismo saber cómo son en teoría a ponerlos en práctica. Ahora que terminamos la asignatura, y con ello las prácticas y nuestro estudio de los métodos ágiles (de momento...), nos paramos a reflexionar sobre las ventajas e inconvenientes que han aportado sobre los métodos tradicionales.

\subsection*{Ventajas}
La ventaja más notable es que, en el mismo tiempo, hemos logrado avanzar más con nuestro proyecto. Si comparamos el grado de desarrollo del mismo en esta asignatura con el de la asignatura Fundamentos de Ingeniería del Software, vemos grandes avances. Durante el desarrollo de la asignatura anterior estuvimos generando grandes cantidades de documentación, diagramas de muchos tipos, ... sin llegar a atisbar realmente cómo sería nuestra aplicación. Esta vez, al contrario, aunque hayamos escrito una documentación mínima sobre el proyecto y los documentos necesarios para planificar el desarrollo del mismo hemos llegado a generar prototipos que nos permiten visualizar cómo sería nuestra aplicación.\\

Hemos vivido de primera mano uno de los objetivos principales de las metodologías ágiles: aceptar el cambio. Los objetivos que habíamos marcado (en forma de historias de usuario) no eran todos los que realmente queríamos para nuestra aplicación. Durante el desarrollo de la primera iteración nos percatamos de algunos detalles o funcionalidad que deseabamos y no estaban en nuestra pila de producto. Consideramos una ventaja usar una metodología que nos permita añadir estas nuevas funcionalidades a la pila de producto en las distintas fases del desarrollo, pues de primeras podemos no acordarnos de todo o no saber exactamente qué es lo que queremos.\\

\subsection*{Inconvenientes}

Sin embargo, no todo fueron cosas buenas. La planificación y estimación de las diferentes tareas requiere un tiempo que fue difícil de ajustar. 

\end{document}
