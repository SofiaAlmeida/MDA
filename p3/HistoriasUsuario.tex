\documentclass[11pt]{article}
\usepackage{fontspec}
\usepackage[spanish]{babel}
%\usepackage[utf8]{inputenc}
\usepackage{listings}
\usepackage{graphicx}
\graphicspath{{../Imagenes/}}

\usepackage{titlepic}

%%% Tablas
\usepackage{tabularx}
\usepackage{float}
\usepackage{adjustbox}
\usepackage{booktabs}
\usepackage{multirow}
\renewcommand{\arraystretch}{1.7}

%%%%%%%%%% Contador HU
\newcounter{HUCounter}
\newcommand{\hu}[1]{\refstepcounter{HUCounter}\textbf{\rmfamily HU-\theHUCounter}\label{#1}}

\begin{document}

\begin{titlepage}
\centering
\vspace{4.5cm}
{\scshape\LARGE Historias de Usuario \par}
\vspace{1.5cm}

\includegraphics[width=10cm]{Logo}

\vspace{3cm}
{\scshape\large \par}
\vspace{1cm}

{Miguel Albertí Pons\\
Sofía Almeida Bruno\\
Pedro Manuel Flores Crespo\\
María Victoria Granados Pozo\\
Lidia Martín Chica
\par}

\end{titlepage}
\newpage

\section{Listado de Historias de Usuario}
En el apartado de prioridad hemos escogido una escala del 1 al 6 siendo el 1: máxima prioridad y el 6: prioridad mínima.
\begin{table}[H]
  \centering
  \begin{tabular}{p{0.1\linewidth}p{0.7\linewidth}p{0.15\linewidth}p{0.15\linewidth}}
    \toprule
    \textbf{Id HU} & \textbf{Título} & \textbf{Estimación} & \textbf{Prioridad}\\
    \midrule
    \hu{} & Un usuario quiere grabar un audioguía sobre su ciudad para compartirla con otros usuarios & 4 & 6\\
    \hu{} & Un usuario quiere escuchar un audioguía sobre la ruta que va a realizar para informarse sobre la ciudad & 3 & 6\\
    \hu{}. & Un usuario quiere ver su historial de viajes guardados/lista de deseos para elegir entre ellos & 4 & 5\\
    \hu{} & Un usuario quiere ver su historial de viajes guardados/lista de deseos para elegir entre ellos & 4 & 5\\
    \hu{} & Un usuario quiere gestionar su maleta. (ver tiempo, tipo de actividad, duración, ver comentarios de otros usuarios, etc). & 4 & 6\\
    \hu{} & Un usuario quiere ofertar un viaje/actividad para compartir su experiencia & & 2\\
    \hu{} & Un usuario quiere valorar un viaje que ha realizado para mostrar su opinión & & 4\\
    \hu{} & Un usuario quiere comentar en una publicación para preguntar una duda o aportar información & & 4\\ 
    \bottomrule
  \end{tabular}
\end{table}

\subsection{Historias que se han dividido en varias}
\begin{table}[H]
  \centering
  \begin{tabular}{p{0.1\linewidth}p{0.7\linewidth}p{0.15\linewidth}p{0.15\linewidth}}
    \toprule
    \textbf{Id HU} & \textbf{Título} & \textbf{Estimación} & \textbf{Prioridad}\\
    \midrule

    \bottomrule
  \end{tabular}
  
\end{table}
\end{document}