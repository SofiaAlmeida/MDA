\documentclass[11pt]{article}
\usepackage{fontspec}
\usepackage[spanish]{babel}
%\usepackage[utf8]{inputenc}
\usepackage{listings}
\usepackage{graphicx}
\graphicspath{{../Imagenes/}}

\usepackage{titlepic}

%%% Tablas
\usepackage{longtable}
\usepackage{tabularx}
\usepackage{float}
\usepackage{adjustbox}
\usepackage{booktabs}
\usepackage{multirow}
\renewcommand{\arraystretch}{1.7}

%%%%%%%%%% Contador HU
\newcounter{HUCounter}
\newcommand{\hu}[1]{\refstepcounter{HUCounter}\textbf{\rmfamily HU-\theHUCounter}\label{#1}}

\begin{document}

\begin{titlepage}
\centering
\vspace{4.5cm}
{\scshape\LARGE Historias de Usuario \par}
\vspace{1.5cm}

\includegraphics[width=10cm]{Logo}

\vspace{3cm}
{\scshape\large \par}
\vspace{1cm}

{Miguel Albertí Pons\\
Sofía Almeida Bruno\\
Pedro Manuel Flores Crespo\\
María Victoria Granados Pozo\\
Lidia Martín Chica
\par}

\end{titlepage}
\newpage

\section{Listado de Historias de Usuario}
En el apartado de prioridad hemos escogido una escala del 1 al 6 siendo el 1: máxima prioridad y el 6: prioridad mínima.
  \begin{longtable}{p{0.12\linewidth}p{0.7\linewidth}p{0.15\linewidth}p{0.15\linewidth}}
    \toprule
    \textbf{Id HU} & \textbf{Título} & \textbf{Estimación} & \textbf{Prioridad}\\
    \midrule
     \hu{} & Un usuario quiere registrarse en la aplicación para poder planificar su viaje & 2 & 1\\
     \hu{} & Un usuario quiere identificarse en la aplicación para  acceder a su contenido & 1 & 1\\
     \hu{} & Un usuario quiere ver las publicaciones para saber cuáles le pueden interesar & 2 & 2\\
    \hu{} & Un usuario quiere subir una publicación para compartir su experiencia & 3 & 2\\
    \hu{} & Un usuario quiere valorar un viaje que ha realizado para mostrar su opinión & 2 & 4\\
    \hu{} & Un usuario quiere comentar en una publicación para preguntar una duda o aportar información & 2 & 4\\ 
    \hu{} & Un usuario quiere ver los comentarios de una publicación para conocer las experiencias y opiniones de otros usuarios & 1 & 4\\ 
    \hu{} & Un usuario quiere ver las valoraciones de una publicación para tenerlo en cuenta a la hora de tomar una decisión & 1 & 4\\ 
    \hu{hu:filtrar} & Un usuario quiere filtrar la búsqueda de las publicaciones para encontrar una que se adapte a sus intereses & - & -\\ 
    \hu{hu:personalizar} & Un usuario quiere personalizar un paquete para adaptarlo a sus gustos & - & - \\ 
    \hu{} & Un usuario quiere seleccionar un viaje/actividad para realizarlo & 3/2 & 2\\ 
    \hu{} & Un usuario quiere guardar un viaje/actividad para verlo en otro momento & 1 & 5\\ 
    \hu{}. & Un usuario quiere ver su historial de viajes guardados/lista de deseos para elegir entre ellos & 1 & 5\\
    \hu{hu:maleta} & Un usuario quiere gestionar su maleta & - & -\\
    \hu{} & Un usuario quiere grabar un audioguía sobre su ciudad para compartirla con otros usuarios & 2 & 6\\
    \hu{} & Un usuario quiere escuchar un audioguía sobre la ruta que va a realizar para informarse sobre la ciudad & 3/2 & 6\\

    
    \bottomrule
\end{longtable}

\subsection{Historias que se han dividido en varias}

  \begin{longtable}{p{0.13\linewidth}p{0.67\linewidth}p{0.15\linewidth}p{0.15\linewidth}}
    \toprule
    \textbf{Id HU} & \textbf{Título} & \textbf{Estimación} & \textbf{Prioridad}\\
    \midrule
    
    \textbf{HU-\ref{hu:filtrar}} & Un usuario quiere filtrar la búsqueda de las publicaciones para encontrar una que se adapte a sus intereses & - & -\\ 
	\textbf{HU-\ref{hu:filtrar}.1} & Un usuario quiere filtrar la búsqueda de las publicaciones por lugar & 1/2 & 3\\ 
	\textbf{HU-\ref{hu:filtrar}.2} & Un usuario quiere filtrar la búsqueda de las publicaciones por fecha & 1/2 & 3\\ 
	\textbf{HU-\ref{hu:filtrar}.3} & Un usuario quiere filtrar la búsqueda de las publicaciones por tipo de publicación (paquete, ruta, actividad, evento) & 1/2 & 3\\ 
	\textbf{HU-\ref{hu:filtrar}.4} & Un usuario quiere filtrar la búsqueda de las publicaciones por tipo de actividad (cultural, deportiva, etc)) & 1/2 & 3\\ 
	\textbf{HU-\ref{hu:filtrar}.5} & Un usuario quiere filtrar la búsqueda de las publicaciones para que aparezcan las más visitadas & 1/2 & 4\\ 
	\textbf{HU-\ref{hu:filtrar}.6} & Un usuario quiere filtrar la búsqueda de las publicaciones para que aparezcan las más realizadas & 1/2 & 4\\ 
	\midrule
	\textbf{HU-\ref{hu:personalizar}} & Un usuario quiere personalizar un paquete para adaptarlo a sus gustos y presupuesto & - & - \\
	\textbf{HU-\ref{hu:personalizar}.1} & Un usuario quiere eliminar una actividad de un paquete & 1/2 & 3 \\ 
	\textbf{HU-\ref{hu:personalizar}.2} & Un usuario quiere añadir una actividad de un paquete & 1 & 3 \\ 
	\textbf{HU-\ref{hu:personalizar}.3} & Un usuario quiere modificar el orden de las actividades de un paquete & 3/2 & 4 \\
	\midrule 
	\textbf{HU-\ref{hu:maleta}} & Un usuario quiere gestionar su maleta & - & -\\
	\textbf{HU-\ref{hu:maleta}.1} & Un usuario quiere obtener una recomendación del vestuario que debe llevar según el pronóstico meteorológico & 5 & 6\\
	\textbf{HU-\ref{hu:maleta}.2} & Un usuario quiere obtener una recomendación del vestuario que debe llevar según el tipo de actividad que va a realizar & 4 & 6\\
	\textbf{HU-\ref{hu:maleta}.3} & Un usuario quiere obtener una recomendación del vestuario que debe llevar según la duración del viaje & 4 & 6\\
    \bottomrule
  \end{longtable}

\section{Cálculo de la velocidad del equipo}
Partimos de un equipo de desarrollo formado  por 5 programadores  que van a dedicar un 60\% de su trabajo al proyecto, ya que el resto del tiempo lo utilizan para avanzar otros proyectos. 

La duración de cada una de las iteraciones que vamos a realizar en el proyecto van a ser de 2 semanas. 
La estimación realizada del esfuerzo de cada una de las historias de usuario se ha expresado en días ideales de programación. En nuestro
entorno de trabajo estimamos que un día ideal de programación se va a corresponder con 2 a 3 días reales de trabajo. 

La duración de una iteración va a ser: 

1 iteración = 2 semanas = 6 días reales 

La velocidad del equipo de desarrollo medido en punto de historia es:  

5 programadores * 6 = 30 días reales por iteración $\Longrightarrow$ de 10 a 15 PH por iteración.  
Se ha decidido usar 12 Puntos de historia como la velocidad estimada del equipo. 

\section{Descripción de las entregas}

Esfuerzo total del proyecto = 40 PH \\
Velocidad del equipo = 12 PH (por iteración) 

En base al esfuerzo necesario y la velocidad estimada del equipo, para el desarrollo del proyecto se van a realizar 4 entregas, cada una de ellas corresponderá a una iteración.
\begin{itemize}
\item \textbf{Entrega 1:} (AQUÍ PONEMOS LA FECHA DE ENTREGA) AQUÍ ESCRIBIRÍAMOS EL OBJETIVO
\end{itemize}

\section{Tarjetas de las HU}
Se incluye una descripción completa de las historias de usuario que se van a tratar en la primera iteración del desarrollo, incluyendo los criterios de aceptación de cada una de ellas.
%%TABLA 
\begin{table}[H]
  \centering
  \begin{tabular}{p{0.3\linewidth}|p{0.7\linewidth}}
    \toprule
    \textbf{Identificador} & \textbf{HU-\ref{hu:}}\\
    \bottomrule
  \end{tabular}

  \begin{tabular}{p{1.028\linewidth}}
    \textbf{Descripción}
    \midrule
  \end{tabular}

  \begin{tabular}{p{0.18\linewidth}|p{0.1\linewidth}|p{0.18\linewidth}|p{0.1\linewidth}|p{0.18\linewidth}|p{0.1\linewidth}}
    \toprule
    \textbf{Estimación} & & \textbf{Prioridad} & & \textbf{Entrega} & \\
    \bottomrule
  \end{tabular}

  \begin{tabular}{p{1.028\linewidth}}
    \textbf{Pruebas de aceptación}\\
  \midrule
  \begin{itemize}
  \item 
  \end{itemize}
\end{tabular}
\begin{tabular}{p{1.028\linewidth}}
  \textbf{Observaciones}\\
  \midrule
\end{tabular}
\end{table}


%%Registro
\begin{table}[H]
  \centering
  \begin{tabular}{p{0.3\linewidth}|p{0.7\linewidth}}
    \toprule
    \textbf{Identificador} & \textbf{HU-\ref{hu:}}\\
    Un usuario quiere registrarse en la aplicación para poder planificar su viaje
    \bottomrule
  \end{tabular}

  \begin{tabular}{p{1.028\linewidth}}
    \textbf{Descripción}
    \midrule
    Un usuario quiere registrarse en la aplicación para poder planificar su viaje. Cuando un usuario quiere registrarse debe introducir su nombre, que debe tener entre 2 y 20 caracteres, sus apellidos que deben tener entre 2 y 50 caracteres, su nombre de usuario que consta de un mínimo de 4 caracteres alfanuméricos y un máximo de 15 caracteres alfanuméricos, la contraseña, que deberá repetir y que consta de un mínimo de 8 caracteres alfanuméricos y un máximo de 15 caracteres alfanuméricos y un e-mail.
  \end{tabular}

  \begin{tabular}{p{0.18\linewidth}|p{0.1\linewidth}|p{0.18\linewidth}|p{0.1\linewidth}|p{0.18\linewidth}|p{0.1\linewidth}}
    \toprule
    \textbf{Estimación} & & \textbf{Prioridad} & & \textbf{Entrega} & \\
    2 && 1 && \\
    \bottomrule
  \end{tabular}

  \begin{tabular}{p{1.028\linewidth}}
    \textbf{Pruebas de aceptación}\\
  \midrule
  \begin{itemize}
  \item No introducir algún campo del registro y comprobar que informa que no se han rellenado todos los campos, ya que son obligatorios
  \item Introducir todos los datos de usuario correctos y comprobar que se crea el usuario en la base de datos con su contraseña asociada y que recibe un e-mail de confirmación
  \item Introducir un nombre con más de 20 caracteres y comprobar que nos informa que el nombre es demasiado largo
  \item Introducir un nombre con menos de 2 caracteres y comprobar que nos informa que el nombre es demasiado corto
  \item Introducir unos apellidos con más de 50 caracteres y comprobar que nos informa que los apellidos son demasiado largos
  \item Introducir unos apellidos con menos de 2 caracteres y comprobar que nos informa que los apellidos son demasiado corto
  \item Introducir un nombre de usuario con más de 15 caracteres y comprobar que nos informa que el nombre de usuario es demasiado largo
  \item Introducir un nombre de usuario con menos de 4 caracteres y comprobar que nos informa que el nombre de usuario es demasiado corto
  \item Introducir un nombre de usuario con caracteres no alfanuméricos y comprobar que nos informa que el nombre de usuario incluye caracteres no permitidos
  \item Introducir un nombre de usuario en formato correcto y existente, con los demás datos introducidos correctamente y comprobar que no se vuelve a crear el usuario y que informa que ya hay un usuario con ese nombre de usuario
  \item Introducir una contraseña con más de 15 caracteres y comprobar que nos informa que la contraseña es demasiado larga
  \item Introducir una contraseña con menos de  8 caracteres y comprobar que nos informa que la contraseña es demasiado corta
  \item Introducir una contraseña en formato correcto y al repetirla escribir una en formato incorrecto y comprobar que informa que la contraseña no cumple el formato
  \item Introducir una contraseña con caracteres no alfanuméricos y comprobar que nos informa que la contraseña incluye caracteres no permitidos
  \item Introducir una contraseña en formato correcto y al repetirla escribir una contraseña distinta en formato correcto y comprobar que informa que las contraseñas no coinciden
  \item Introducir contraseña/nombre usuario incorrectos varias veces. A la tercera indica que no podemos volver a intentarlo hasta dentro de cierto tiempo
  \item Introducir un e-mail en formato correcto y que no existe y comprobar que al intentar enviar el correo, al no existir no lo envía y que informa que el e-mail introducido no existe
  \item Introducir un e-mail en formato no correcto y que por tanto, no existe y comprobar informa que el e-mail introducido no es correcto
  \end{itemize}
\end{tabular}
\begin{tabular}{p{1.028\linewidth}}
  \textbf{Observaciones}\\
  \midrule
\end{tabular}
\end{table}

%%Identificación 
\begin{table}[H]
  \centering
  \begin{tabular}{p{0.3\linewidth}|p{0.7\linewidth}}
    \toprule
    \textbf{Identificador} & \textbf{HU-\ref{hu:}}\\
    Un usuario quiere identificarse en la aplicación para acceder a su contenido
    \bottomrule
  \end{tabular}

  \begin{tabular}{p{1.028\linewidth}}
    \textbf{Descripción}
    \midrule
    Un usuario quiere identificarse en la aplicación para acceder a su contenido. Para iniciar sesión es necesario introducir el nombre de usuario y la contraseña. El nombre de usuario consta de un mínimo de 4 caracteres alfanuméricos y un máximo de 15 caracteres alfanuméricos. La contraseña consta de un mínimo de 8 caracteres alfanuméricos y un máximo de 15 caracteres alfanuméricos.
  \end{tabular}

  \begin{tabular}{p{0.18\linewidth}|p{0.1\linewidth}|p{0.18\linewidth}|p{0.1\linewidth}|p{0.18\linewidth}|p{0.1\linewidth}}
    \toprule
    \textbf{Estimación} & & \textbf{Prioridad} & & \textbf{Entrega} & \\
    1 && 1 && \\
    \bottomrule
  \end{tabular}

  \begin{tabular}{p{1.028\linewidth}}
    \textbf{Pruebas de aceptación}\\
  \midrule
  \begin{itemize}
  \item Comprobar que la contraseña no se muestra al escribirla
  \item Introducir un nombre de usuario existente con su contraseña correcta y comprobar que se inicia la sesión con ese usuario
  \item Introducir un nombre de usuario con más de 15 caracteres y la contraseña correcta y comprobar que nos informa que no podemos acceder
  \item Introducir un nombre de usuario con menos de 4 caracteres y la contraseña correcta y comprobar que nos informa que no podemos acceder
  \item Introducir un nombre de usuario con una longitud entre 4 y 15 caracteres pero no todos alfanuméricos y la contraseña correcta y comprobar que nos informa que no podemos acceder
  \item Introducir un nombre de usuario correcto y una contraseña incorrecta, una que no cumpla restricciones y comprobar que nos informa que no podemos entrar
  \item Introducir un nombre de usuario correcto y una contraseña incorrecta, que cumpla restricciones pero no coincida con la asociada al usuario en la base de datos y comprobar que nos informa que no podemos entrar
  \item Introducir un nombre de usuario que cumpla las restricciones pero no esté en la base de datos y contraseña correcta en formato y comprobar que no permite entrar en el sistema
  \item No introducir nombre de usuario y ni contraseña e intentar entrar en el sistema, no debe permitirlo
  \item Introducir nombre de usuario correcto y no introducir contraseña. Nos debe indicar que no hemos introducido contraseña
  \item Introducir una contraseña correcta y no introducir un nombre de usuario. Nos debe indicar que no hemos introducido nombre de usuario
  \item Introducir contraseña/nombre usuario incorrectos varias veces. A la tercera indica que no podemos volver a intentarlo hasta dentro de cierto tiempo
  
  \end{itemize}
\end{tabular}
\begin{tabular}{p{1.028\linewidth}}
  \textbf{Observaciones}\\
  \midrule
\end{tabular}
\end{table}




\end{document}
