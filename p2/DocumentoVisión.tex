\documentclass[11pt]{article}
\usepackage[spanish]{babel}
\usepackage[utf8]{inputenc}
\usepackage{listings}
\usepackage{graphicx}
\graphicspath{{../Imagenes/}}

\usepackage[paper=portrait, pagesize]{typearea}
\usepackage{titlepic}

%%% Tablas
\usepackage{tabularx}
\usepackage{float}
\usepackage{adjustbox}
\usepackage{booktabs}
\usepackage{multirow}
\renewcommand{\arraystretch}{1.7}

\begin{document}

\begin{titlepage}
\centering
\vspace{4.5cm}
{\scshape\LARGE Documento de Visión:\\MakeATravel \par}
\vspace{1.5cm}

\includegraphics[width=16cm]{Logo}

\vspace{3cm}
{\scshape\large \par}
\vspace{1cm}

{Miguel Albertí Pons\\
Sofía Almeida Bruno\\
Pedro Manuel Flores Crespo\\
María Victoria Granados Pozo\\
Lidia Martín Chica
\par}

\end{titlepage}
\newpage

\section*{Historial de Cambios}
\begin{table}[H]
  \centering
  \begin{tabular}{p{0.15\linewidth}p{0.1\linewidth}p{0.4\linewidth}p{0.25\linewidth}}
    \toprule
    \textbf{Fecha} & \textbf{Versión} & \textbf{Descripción} & \textbf{Autor}\\
    07/Oct/2019 & 1.0 & Propuesta inicial del documento & MVG LMC MAP\\
    \midrule
    \bottomrule
  \end{tabular}
\end{table}
\newpage

\tableofcontents
\newpage
%%%%%%%%%%%%%%%%%%%%%%%%%%%%%%%%%%%%%%%%%%%%%%%%%%%%%%%%%%%%%%
%%%%%                   INTRODUCCIÓN                      %%%%
%%%%%%%%%%%%%%%%%%%%%%%%%%%%%%%%%%%%%%%%%%%%%%%%%%%%%%%%%%%%%%

\section{Introducción}
\subsection{Propósito}
Este documento proporciona una visión general sobre la aplicación MakeATravel, recogiendo las necesidades y características del sistema de gestión de actividades y viajes. Así como algunas funcionalidades opcinales como consejos para hacer la maleta, dependiendo del tiempo que vaya a hacer en el destino o la duración del viaje, y mapas de las ciudades con explicaciones de las mismas.

El documento se centra en definir el alcance, oportunidades de negocio, identificar los stakeholders y sus perfiles para poder llevar a cabo un desarrollo ajustado de la aplicación.

Además, encontraremos un apartado donde definiremos detalladamente el producto (sección \ref{4}) que será desarrollado para empresa Platilla.


\subsection{Alcance}
Nuestro producto constará de un entregable, una aplicación movil, dicha aplicación servirá para planificar viajes y actividades telemáticamente.

Esta app la encontraremos en los sistemas operativos de Android e iOS, para su descarga habrá que acceder a Play Store o a App Store. 

%%COMPLETAR
\subsection{Definición, Acrónimos y Abreviaciones}

UOA: Usuarios que ofrecen actividades\\

USEA: Usuarios que seleccionan actividades\\

UC: Usuarios que comentan.

\subsection{Resumen}
Actualmente, viajar está en auge y los paquetes de las compañías están anticuados y poco personalizados. Además, si no quieres acudir a una agencia de viajes, ya que podría incrementar el precio, o dedicar muchas horas a organizar un viaje que se adapte a ti, con esta app pretendemos que a través de las experiencia vividas por otros usuarios puedas planear tu viaje ideal sin salir de la aplicación.

En esta aplicación los usuarios tienen un papel principal, en primer lugar serán ellos los que \textbf{ofertan} sus experiencias y sus actividades, mediante fotos y descripciones, que otros usuarios podrán \textbf{valorar, comentar, seleccionar y personalizar} el viaje.

Las publicaciones de los usuarios podrán ser de distintos tipos como \textbf{rutas} (gastronómicas, turísticas, ...), \textbf{eventos} (musicales, obras de teatro, ...) y \textbf{actividades} (deportivas, culturales, ...). Asímismo un usuario puede crear un \textbf{paquete} en el que agrupe varias publicaciones o seleccionar paquetes hechos y adaptarlos a sus intereses.

Los principales objetivos que queremos cubrir son: la \textbf{facilidad} a la hora de organizar un viaje usando filtros que se amolden a lo que buscamos y basándonos en los comentarios y valoraciones de otros usuarios; \textbf{beneficiar económicamente} a aquellos usuarios cuyas publicaciones sean las mejor valoradas; el \textbf{acceso} a todo tipo de usuarios (con discapacidad, de cualquier edad, ...) a usar la aplicación.

Otras funcionalidades que se podrían incorporar son: \textbf{audioguías} para las actividades culturales; recomendaciones para \textbf{hacer la maleta} según las actividades que se vayan a realizar, el destino o el tiempo.

\newpage

%%%%%%%%%%%%%%%%%%%%%%%%%%%%%%%%%%%%%%%%%%%%%%%%%%%%%%%%%%%%%%
%%%%%                   POSICIONAMIENTO                   %%%%
%%%%%%%%%%%%%%%%%%%%%%%%%%%%%%%%%%%%%%%%%%%%%%%%%%%%%%%%%%%%%%
\section{Posicionamiento}
\subsection{Oportunidad de Negocio}
Actualmente si se quiere organizar un viaje o buscar una actividad,  requiere mucho tiempo y la visita de muchas páginas web para poder tener un viaje personalizado, o bien acudir a una agencia de viajes y pagar un precio excesivo para lo que te ofrecen.

Con MakeATravel estos dos problemas son inexistentes, en una misma aplicación tendrás miles de opciones donde elegir y pagar con un solo click, sin ningún coste extra.

\subsection{Declaración del Problema}

\begin{table}[H]
  \centering
  \begin{tabular}{p{0.35\linewidth}|p{0.65\linewidth}}
    \toprule
    \textbf{El problema de} & Tener que visitar muchas páginas web
    
    Elevados costes de las agencias de viajes
    
    Desconocer empresas locales y pequeñas, que ofrecen actividades más económicas por la baja demanda\\
    \textbf{Afecta} & Usuarios finales 
    
    Pequeño comercio\\
    \textbf{El impacto del problema es} & Organizar tu viaje con total comodidad, ofreciendo actividades únicas a través de las experiencias de otras personas\\
    \textbf{Una solución exitosa sería} & Encontrar viajes más baratos de los que puedes encontrar en el mercado actualmente. Además de ayudar a la promoción de pequeñas empresas que oferten actividades\\
    \bottomrule
  \end{tabular}
\end{table}

\subsection{Solución propuesta}
Para los usuarios finales que quieran organizar un viaje y se encuentran con las opciones de visitar muchas páginas web o pagar excesivos costes en agencias de viajes una opción sería utilizar una única aplicación móvil en la que puedan encontrar agrupada toda la información que necesiten en base a las experiencias y opiniones de otros usuarios. La aplicación a desarrollar, además de esto, permite la personalización de dichos viajes y actividades desde una interfaz gráfica sencilla disponible siempre que se tenga conexión a internet.

Para los pequeños comercios que resultan desconocidos para los turistas y que por este hecho ofertan actividades a menor precio nuestra apliación les permitirá darse a conocer. Lo harán mediante las publicaciones de usuarios que sí se hayan detenido a encontrar estas empresas organizadoras de actividades.
\newpage

%%%%%%%%%%%%%%%%%%%%%%%%%%%%%%%%%%%%%%%%%%%%%%%%%%%%%%%%%%%%%%
%%%%%             USUARIOS Y STAKEHOLDERS                 %%%%
%%%%%%%%%%%%%%%%%%%%%%%%%%%%%%%%%%%%%%%%%%%%%%%%%%%%%%%%%%%%%%
\section{Descripción de Usuarios y Stakeholders}
[A fin de ejecutar eficazmente los productos y servicios que satisfagan las necesidades de las partes interesadas en este proyecto, es necesario conocer quiénes son los interesados y hacer que participen en el proceso de desarrollo. Un grupo importante de actores está formado por los usuarios (finales) del sistema.
Esta sección proporciona una visión general de las partes interesadas, sus intereses y los problemas que se resuelven por la solución propuesta.]

\subsection{Resumen de los stakeholders}
\begin{table}[H]
  \centering
  \begin{tabular}{p{0.2\linewidth}p{0.4\linewidth}p{0.4\linewidth}}
    \toprule
    \textbf{Nombre} & \textbf{Descripción} & \textbf{Responsabilidades} \\
    \midrule
    Lidia Martín & Coordinadora del proyecto & Asegurarse que la aplicación cumpla los plazo de entrega\\
    María Victoria Granados & Catalogadora & Encargada de la documentación del proyecto\\
    Miguel Albertí & Moderador & Asegurar del correcto funcionamiento del equipo\\
    Sofía Almeida & Presentadora & Encargada del marketing de la aplicación\\
    Pedro Manuel Flores & Gestor de Calidad & Asegurarse de que el producto cumpla los estándares\\
    \bottomrule
  \end{tabular}
\end{table}

\subsection{Resumen de los usuarios}
\begin{table}[H]
  \centering
  \begin{tabular}{p{0.2\linewidth}p{0.4\linewidth}p{0.4\linewidth}}
    \toprule
    \textbf{Nombre} & \textbf{Descripción} & \textbf{Stakeholder} \\
    \midrule
    ACT\_1 UOA & Usuarios que ofertan viajes o actividades & STK1 Ofertar\\
    ACT\_2 USEA & Usuarios que seleccionan una actividad o viaje, que haya realizado el usuario ACT\_1 & STK2 Seleccionar\\
    ACT\_3 UC & Usuarios que comentan los post, ya sea por que tengan dudas sobre el paquete de viaje o por que la han realizado y desean dejar su experiencia & STK3 Comentar/Valorar\\
    ACT\_4 UV & Usuarios que valoran una actividad o viaje. Para ello deberán de haberla realizado. & STK3 Comentar/Valorar\\
    ACT\_5  Empleado de Marketing & Responsable de ofertas de lanzamiento, publicidad y otros aspectos relacionados con el marketing & STK4 Marketing\\
    \bottomrule
  \end{tabular}
\end{table}

\subsection{Perfil de los Stakeholders}

%PERFIL1
\begin{table}[H]
  \centering
  \begin{tabular}{p{0.35\linewidth}|p{0.65\linewidth}}
    \toprule
       \textbf{Descripción} & Empresas del sector, este stakeholder son reticentes a la hora de conocer nuestra aplicación, dentro de este grupo el mayor exponente son las empresas de viaje.\\ 
    \textbf{Responsabilidades} & Este stakeholder no tiene ninguna responsabilidad, aunque intentarán espfrzarse para mejorar sus paquetes de viajes y actividades para que nuestra app no tenga lugar en el mercado. \\
    \textbf{Criterios de éxito} & Las empresas competidoras podrán decir que tienen éxito en el momento que nuestra app salga al mercado y ellos no piedan a sus clientes. \\
    \bottomrule
  \end{tabular}
\end{table}

%PERFIL2
\begin{table}[H]
  \centering
  \begin{tabular}{p{0.35\linewidth}|p{0.65\linewidth}}
    \toprule
    \textbf{Descripción} & Los programadores son el grupo encargado de realizar la parte técnica de la aplicación, es decir, tanto el frontend como el backend, aunque dentro de estas dos categorias encontramos varios grupos su éxito y responsabilidades son parecidas\\ 
    \textbf{Responsabilidades} & Este stakeholder se encargará de programar todas las funcionalidades que se hayan definido. \\
    \textbf{Criterios de éxito} & Podremos asegurar que han cubierto sus responsabilidades una vez se puedan usar todas las funcionalidades de la aplicación sin ningún problema. \\
    \bottomrule
  \end{tabular}
\end{table}

%PERFIL3
\begin{table}[H]
  \centering
  \begin{tabular}{p{0.35\linewidth}|p{0.65\linewidth}}
    \toprule
    \textbf{Descripción} & Departamento de Markting.\\ 
    \textbf{Responsabilidades} & Este conjunto de usuarios tiene la misión de llevar a cabo una buena campaña publicitaría del producto \\
    \textbf{Criterios de éxito} & Su éxito se medirá una vez salga el producto, si en un breve período de tiempo un gran número de personas estan registradas y usan la app querra decir que la camapaña causo efecto sobre los usuarios. \\
    \bottomrule
  \end{tabular}
\end{table}



\subsection{Perfil de los Usuarios}
%STK1
\begin{table}[H]
  \centering
  \begin{tabular}{p{0.35\linewidth}|p{0.65\linewidth}}
    \toprule
    \textbf{Descripción} & STK1 Ofertar\\
    \textbf{Responsabilidades} & Usuarios que durante la realización del viaje o al finalizar, publican los lugares que han visitado, con fotos e indicaciones de los lugares\\
    \textbf{Criterios de éxito} &  \\
    \bottomrule
  \end{tabular}
\end{table}

%STK2
\begin{table}[H]
  \centering
  \begin{tabular}{p{0.35\linewidth}|p{0.65\linewidth}}
    \toprule
    \textbf{Descripción} & STK2 Seleccionar\\
    \textbf{Responsabilidades} & Usuarios que seleccionan las publicaciones que se obtienen tras una filtración según lo que busque el usuario. Estas publicaciones son las de otros usuarios que ya han realizado la ruta, el viaje o la actividad\\
    \textbf{Criterios de éxito} &  \\
    \bottomrule
  \end{tabular}
\end{table}

%STK3.1
\begin{table}[H]
  \centering
  \begin{tabular}{p{0.35\linewidth}|p{0.65\linewidth}}
    \toprule
    \textbf{Descripción} & STK3 Comentar\\
    \textbf{Responsabilidades} & Usuarios que comentan las publicaciones, para recomendarla o para preguntar las dudas que tengan sobre la misma.\\
    \textbf{Criterios de éxito} &  \\
    \bottomrule
  \end{tabular}
\end{table}

%STK3.2
\begin{table}[H]
  \centering
  \begin{tabular}{p{0.35\linewidth}|p{0.65\linewidth}}
    \toprule
    \textbf{Descripción} & STK3 Valorar\\
    \textbf{Responsabilidades} & Usuarios que tras la realización de la ruta o actividad, valoran la actividad de 1 a 5. Si no se realiza no se podrá valorar\\
    \textbf{Criterios de éxito} &  \\
    \bottomrule
  \end{tabular}
\end{table}

%STK4
\begin{table}[H]
  \centering
  \begin{tabular}{p{0.35\linewidth}|p{0.65\linewidth}}
    \toprule
    \textbf{Descripción} & STK4 Marketing\\
    \textbf{Responsabilidades} & Responsable de la publicidad de la aplicación.\\
    \textbf{Criterios de éxito} &  \\
    \bottomrule
  \end{tabular}
\end{table}




\subsection{Necesidades clave de los Stakeholders o usuarios}
[Proporcionar las necesidades o problemas percibidos por los diferentes roles de los stakeholders. Esta visión general se utiliza a continuación para dar prioridad a las necesidades.]
          
\begin{table}[H]
  \centering
  \begin{tabular}{p{0.15\linewidth}p{0.15\linewidth}p{0.15\linewidth}p{0.25\linewidth}p{0.25\linewidth}}
    \toprule
    \textbf{Necesidad} & \textbf{Prioridad} & \textbf{Incumbe} & \textbf{Solución Actual} & \textbf{Solución propuesta} \\
    \midrule
    \bottomrule
  \end{tabular}
\end{table}
\newpage

%%%%%%%%%%%%%%%%%%%%%%%%%%%%%%%%%%%%%%%%%%%%%%%%%%%%%%%%%%%%%%
%%%%%               RESUMEN DEL PRODUCTO                  %%%%
%%%%%%%%%%%%%%%%%%%%%%%%%%%%%%%%%%%%%%%%%%%%%%%%%%%%%%%%%%%%%%
\section{Resumen del producto}\label{4}
\subsection{Perspectiva del producto}
[Describir el producto en relación a otros productos del entorno del cliente. Si el producto es independiente, indicarlo aquí. Si el producto es un componente de un sistema más grande o se conecta a otros sistemas, mostrar cómo estos sistemas interactúan entre si. Se puede facilitar un diagrama para mostrar las relaciones.]


\subsection{Resumen de capacidades}
[Proporcionar una lista de características y dar prioridad a la lista. Se puede utilizar parte del modelo de casos de uso cómo descripción de este apartado. La lista de características o el modelo de casos de uso proporcionan la base para describir el alcance del proyecto.]

\begin{table}[H]
  \centering
  \begin{tabular}{p{0.4\linewidth}p{0.6\linewidth}}
    \toprule
    \textbf{Beneficion del Cliente} & \textbf{Características de Soporte} \\
    \midrule
    \bottomrule
  \end{tabular}
\end{table}

\subsection{Supuestos y dependencias}

Para cumplir unos estandares de desarrollo de software utilizaremos la ISO 9001, además para la realización se este proyecto se ha asumido que la app  se utilizará en los sistemas IOS o android, por otro lado también se ha supuesto que la mayor parte de usuarios estarán entre los 20-50 años, hecho que hará que no haya ningún tutorial extenso, únicamente contará con una pequeña guía.

Dependeremos de páginas web o aplicaciones externas para realizar los pagos y las reservas, la aplicación únicamente vinculará publicaciones de los usuarios con las páginas oficiales. 

\end{document}
