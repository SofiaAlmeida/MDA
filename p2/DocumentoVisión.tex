\documentclass[11pt]{article}
\usepackage[spanish]{babel}
\usepackage[utf8]{inputenc}
\usepackage{listings}
\usepackage{graphicx}
\graphicspath{{../Imagenes/}}

\usepackage[paper=portrait, pagesize]{typearea}
\usepackage{titlepic}

%%% Tablas
\usepackage{tabularx}
\usepackage{float}
\usepackage{adjustbox}
\usepackage{booktabs}
\usepackage{multirow}
\renewcommand{\arraystretch}{1.7}

\begin{document}

\begin{titlepage}
\centering
\vspace{4.5cm}
{\scshape\LARGE Documento de Visión:\\MakeATravel \par}
\vspace{1.5cm}

\includegraphics[width=16cm]{Logo}

\vspace{3cm}
{\scshape\large \par}
\vspace{1cm}

{Miguel Albertí Pons\\
Sofía Almeida Bruno\\
Pedro Manuel Flores Crespo\\
María Victoria Granados Pozo\\
Lidia Martín Chica
\par}

\end{titlepage}
\newpage

\section*{Historial de Cambios}
\begin{table}[H]
  \centering
  \begin{tabular}{p{0.1\linewidth}p{0.15\linewidth}p{0.4\linewidth}p{0.25\linewidth}}
    \toprule
    \textbf{Fecha} & \textbf{Versión} & \textbf{Descripción} & \textbf{Autor}\\
    07/Oct/2019 & 1.0 & Propuesta inicial del documento & MVG LMC MAP\\
    \midrule
    \bottomrule
  \end{tabular}
\end{table}
\newpage

\tableofcontents
\newpage
%%%%%%%%%%%%%%%%%%%%%%%%%%%%%%%%%%%%%%%%%%%%%%%%%%%%%%%%%%%%%%
%%%%%                   INTRODUCCIÓN                      %%%%
%%%%%%%%%%%%%%%%%%%%%%%%%%%%%%%%%%%%%%%%%%%%%%%%%%%%%%%%%%%%%%

\section{Introducción}
\subsection{Propósito}
Este documento proporciona una visión general sobre la aplicación MakeATravel, recogiendo las necesidades y características del sistema de gestión de actividades y viajes. Esta aplicación se desarrollará para empresa Platilla.

El documento se centra en definir el alcance, oportunidades de negocio, identificar los stakeholders y sus perfiles para poder llevar a cabo un desarrollo ajustado de la aplicación.

Además encontraremos un apartado donde definiremos detalladamente el producto (punto 4).


\subsection{Alcance}
Nuestro producto constará de un entregable, una aplicación movil, dicha aplicación servirá para planificar viajes y actividades telemáticamente.

Esta app la encontraremos en los sistemas operativos de Android e iOS, para su descarga habrá que acceder a Play Store o a App Store. 

%%COMPLETAR
\subsection{Definición, Acrónimos y Abreviaciones}
UOA : Usuarios que ofrecen actividades\\

USEA : Usuarios que seleccionan actividades 

\subsection{Resumen}
\newpage

%%%%%%%%%%%%%%%%%%%%%%%%%%%%%%%%%%%%%%%%%%%%%%%%%%%%%%%%%%%%%%
%%%%%                   POSICIONAMIENTO                   %%%%
%%%%%%%%%%%%%%%%%%%%%%%%%%%%%%%%%%%%%%%%%%%%%%%%%%%%%%%%%%%%%%
\section{Posicionamiento}
\subsection{Oportunidad de Negocio}
Actualmente si se quiere organizar un viaje o buscar una actividad,  requiere mucho tiempo y la visita de muchas web para poder tener un viaje personalizado, o bien acudir a una agencia de viajes y pagar un precio excesivo para lo que te ofrecen.

Con MakeATravel estos dos problemas son inexistentes, en una misma aplicación tendrás miles de opciones donde elegir y pagar con un solo click, sin ningún coste extra.


\subsection{Declaración del Problema}
[Describa el problema que afecta a los stakeholders interesados, el impacto que tendria en ellos y lo que le beneficiaría una solución. Se puede usar la siguiente tabla:]
El siguiente cuadro detalla el problema así como los afectados por este.

\begin{table}[H]
  \centering
  \begin{tabular}{p{0.35\linewidth}|p{0.65\linewidth}}
    \toprule
    \textbf{El problema de} & Tener que visitar muchas páginas web
    
    Elevados costes de las agencias de viajes
    
    Desconocer empresas locales y pequeñas, que ofrecen actividades más económicas por la baja demanda\\
    \textbf{Afecta} & Usuarios finales 
    
    Pequeño comercio\\
    \textbf{El impacto del problema es} & Organizar tu viaje con total comodidad, ofreciendo actividades únicas a través de las experiencias de otras personas.\\
    \textbf{Una solución exitosa sería} & Encontrar viajes más baratos de los que puedes encontrar en el mercado actualmente. Además de ayudar a la promoción de pequeñas empresas que oferten actividades\\
    \bottomrule
  \end{tabular}
\end{table}

\subsection{Solución propuesta}
[Proporcionar posibles soluciones alternativas que las partes interesadas consideran viables. Enumere estas alternativas, agrupadas por las partes interesadas, y proporcionar una visión general de su fuerza y su debilidad (desde la perspectiva de que las partes interesadas)].

RELLENAAAR

\newpage

%%%%%%%%%%%%%%%%%%%%%%%%%%%%%%%%%%%%%%%%%%%%%%%%%%%%%%%%%%%%%%
%%%%%             USUARIOS Y STAKEHOLDERS                 %%%%
%%%%%%%%%%%%%%%%%%%%%%%%%%%%%%%%%%%%%%%%%%%%%%%%%%%%%%%%%%%%%%
\section{Descripción de Usuarios y Stakeholders}
[A fin de ejecutar eficazmente los productos y servicios que satisfagan las necesidades de las partes interesadas en este proyecto, es necesario conocer quiénes son los interesados y hacer que participen en el proceso de desarrollo. Un grupo importante de actores está formado por los usuarios (finales) del sistema.
Esta sección proporciona una visión general de las partes interesadas, sus intereses y los problemas que se resuelven por la solución propuesta.]

\subsection{Resumen de los stakeholders}
\begin{table}[H]
  \centering
  \begin{tabular}{p{0.2\linewidth}p{0.4\linewidth}p{0.4\linewidth}}
    \toprule
    \textbf{Nombre} & \textbf{Descripción} & \textbf{Responsabilidades} \\
    \midrule
    Lidia Martín & Coordinadora del proyecto & Asegurarse que la aplicación cumpla los plazo de entrega\\
    María Victoria Granados & Catalogadora & Encargada de la documentación del proyecto\\
    Miguel Albertí & Moderador & Asegurar del correcto funcionamiento del equipo\\
    Sofía Almeida & Presentadora & Encargada del marketing de la aplicación\\
    Pedro Manuel Flores & Gestor de Calidad & Asegurarse de que el producto cumpla los estándares\\
    \bottomrule
  \end{tabular}
\end{table}

\subsection{Resumen de los usuarios}
\begin{table}[H]
  \centering
  \begin{tabular}{p{0.2\linewidth}p{0.4\linewidth}p{0.4\linewidth}}
    \toprule
    \textbf{Nombre} & \textbf{Descripción} & \textbf{Stakeholder} \\
    ACT\_1 UOA & Usuarios que ofertan viajes o actividades & STK1 UOA\\
    ACT\_2 USEA & Usuarios que seleccionan una actividad o viaje, que haya realizado el usuario ACT\_1 & STK2 USEA\\
    ACT\_3  Empleado de Marketing& Responsable de ofertas de lanzamiento, publicidad, política de ventas y otros aspectos relacionados con el marketing &\\
    \midrule
    \bottomrule
  \end{tabular}
\end{table}

\subsection{Perfil de los Stakeholders}


\begin{table}[H]
  \centering
  \begin{tabular}{p{0.35\linewidth}|p{0.65\linewidth}}
    \toprule
    \textbf{Descripción} & Empresas del sector, este stakeholder son reticentes a la hora de conocer nuestra aplicación, dentro de este grupo el mayor exponente son las empresas de viaje.\\ 
    \textbf{Responsabilidades} & Este stakeholder no tiene ninguna responsabilidad, aunque intentarán espfrzarse para mejorar sus paquetes de viajes y actividades para que nuestra app no tenga lugar en el mercado. \\
    \textbf{Criterios de éxito} & Las empresas competidoras podrán decir que tienen éxito en el momento que nuestra app salga al mercado y ellos no piedan a sus clientes. \\
    \bottomrule
  \end{tabular}
\end{table}

\begin{table}[H]
  \centering
  \begin{tabular}{p{0.35\linewidth}|p{0.65\linewidth}}
    \toprule
    \textbf{Descripción} & Los programadores son el grupo encargado de realizar la parte técnica de la aplicación, es decir, tanto el frontend como el backend, aunque dentro de estas dos categorias encontramos varios grupos su éxito y responsabilidades son parecidas\\ 
    \textbf{Responsabilidades} & Este stakeholder se encargará de programar todas las funcionalidades que se hayan definido. \\
    \textbf{Criterios de éxito} & Podremos asegurar que han cubierto sus responsabilidades una vez se puedan usar todas las funcionalidades de la aplicación sin ningún problema. \\
    \bottomrule
  \end{tabular}
\end{table}

\begin{table}[H]
  \centering
  \begin{tabular}{p{0.35\linewidth}|p{0.65\linewidth}}
    \toprule
    \textbf{Descripción} & Departamento de Markting.\\ 
    \textbf{Responsabilidades} & Este conjunto de usuarios tiene la misión de llevar a cabo una buena campaña publicitaría del producto \\
    \textbf{Criterios de éxito} & Su éxito se medirá una vez salga el producto, si en un breve período de tiempo un gran número de personas estan registradas y usan la app querra decir que la camapaña causo efecto sobre los usuarios. \\
    \bottomrule
  \end{tabular}
\end{table}

\subsection{Perfil de los Usuarios}
\begin{table}[H]
  \centering
  \begin{tabular}{p{0.35\linewidth}|p{0.65\linewidth}}
    \toprule
    \textbf{Descripción} &\\
    \textbf{Responsabilidades} &  \\
    \textbf{Criterios de éxito} &  \\
    \bottomrule
  \end{tabular}
\end{table}

\subsection{Necesidades clave de los Stakeholders o usuarios}
[Proporcionar las necesidades o problemas percibidos por los diferentes roles de los stakeholders. Esta visión general se utiliza a continuación para dar prioridad a las necesidades.]
          
\begin{table}[H]
  \centering
  \begin{tabular}{p{0.15\linewidth}p{0.15\linewidth}p{0.15\linewidth}p{0.25\linewidth}p{0.25\linewidth}}
    \toprule
    \textbf{Necesidad} & \textbf{Prioridad} & \textbf{Incumbe} & \textbf{Solución Actual} & \textbf{Solución propuesta} \\
    \midrule
    \bottomrule
  \end{tabular}
\end{table}
\newpage

%%%%%%%%%%%%%%%%%%%%%%%%%%%%%%%%%%%%%%%%%%%%%%%%%%%%%%%%%%%%%%
%%%%%               RESUMEN DEL PRODUCTO                  %%%%
%%%%%%%%%%%%%%%%%%%%%%%%%%%%%%%%%%%%%%%%%%%%%%%%%%%%%%%%%%%%%%
\section{Resumen del producto}
\subsection{Perspectiva del producto}
 El sistema a desarrollar es una aplicación para la empresa \textit{Platilla}. Con ella se podrán seleccionar, ofertar y valorar tanto rutas de viajes como actividades culturales.


\subsection{Resumen de capacidades}
A continuación se muestra una lista con los principales beneficios que obtendrá el cliente:

\begin{table}[H]
  \centering
  \begin{tabular}{p{0.4\linewidth}p{0.6\linewidth}}
    \toprule
    \textbf{Beneficion del Cliente} & \textbf{Características de Soporte} \\
     Facilidad en la búsqueda de actividades y rutas & Mecanismo de búsqueda con diferentes filtros \\
     Personalización & Posibilidad de edición de los paquetes ofertados por la aplicación \\
     Conocer opinión de otros usuarios & Sección específica con comentarios de otras personas y valoraciones\\
     Benefio económico & Sistema de recompensas para los usuarios que suban las actividades/rutas más realizadas \\
    \midrule
    \bottomrule
  \end{tabular}
\end{table}

\subsection{Supuestos y dependencias}


Para cumplir unos estandares de desarrollo de software utilizaremos la ISO 9001, además para la realización se este proyecto se ha asumido que la app  se utilizará en los sistemas IOS o android, por otro lado también se ha supuesto que la mayor parte de usuarios estarán entre los 20-50 años, hecho que hará que no haya ningún tutorial extenso, únicamente contará con una pequeña guía.

Dependeremos de páginas web o aplicaciones externas para realizar los pagos y las reservas, la aplicación únicamente vinculará publicaciones de los usuarios con las páginas oficiales. 

\end{document}
