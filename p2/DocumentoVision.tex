\documentclass[11pt]{article}
\usepackage[spanish]{babel}
\usepackage[utf8]{inputenc}
\usepackage{listings}
\usepackage{graphicx}
\graphicspath{{../Imagenes/}}

\usepackage{titlepic}

%%% Tablas
\usepackage{tabularx}
\usepackage{float}
\usepackage{adjustbox}
\usepackage{booktabs}
\usepackage{multirow}
\renewcommand{\arraystretch}{1.7}

%Índice
\addtocontents{toc}{\hspace{-7.5mm} \textbf{Secciones}}
\addtocontents{toc}{\hfill \textbf{Página} \par}
\addtocontents{toc}{\vspace{-2mm} \hspace{-7.5mm} \hrule \par}


\begin{document}

\begin{titlepage}
\centering
\vspace{4.5cm}
{\scshape\LARGE Documento de Visión \par}
\vspace{1.5cm}

\includegraphics[width=16cm]{Logo}

\vspace{3cm}
{\scshape\large \par}
\vspace{1cm}

{Miguel Albertí Pons\\
Sofía Almeida Bruno\\
Pedro Manuel Flores Crespo\\
María Victoria Granados Pozo\\
Lidia Martín Chica
\par}

\end{titlepage}

\section{Historial de Cambios}
\begin{table}[H]
	\centering
	\begin{tabular*}{0.75\textwidth}{c c c c}
		\textbf{Fecha} & \textbf{Versión} & \textbf{Descripción} & \textbf{Autor} \\ \hline
		&  &  &  \\ \hline
		&  &  &  \\ \hline
		&  &  &  \\ \hline
	\end{tabular*}
\end{table}

\newpage
\thispagestyle{empty}
\tableofcontents
\newpage

\section{Introducción}
\subsection{Propósito}
%[Definir el propósito de este documento de Visión. Podría ser algo como:]
%Este documento proporciona una visión sobre el proyecto <Nombre> desarrollado para <Empresa>.

\subsection{Alcance}
%[Definir una descripcion corta del alcance del software]

\subsection{Definición, Acrónimos y Abreviaciones}

\subsection{Resumen}

\newpage
\section{Posicionamiento}
\subsection{Oportunidad de Negocio}
%[Dar una descripción concisa de los procesos de negocio actuales, que deje claro por qué la aplicación prevista sería una solución para el problema planteado. Prestar atención al entorno de trabajo de los usuarios finales]

\subsection{Declaración del Problema}
%[Describa el problema que afecta a los stakeholders interesados, el impacto que tendria en ellos y lo que le beneficiaría una solución. Se puede usar la siguiente tabla:]
\begin{table}[H]
	\centering
	\begin{tabular*}{0.75\textwidth}{c | c}
		\hline
		\textbf{El problema de} & \\ \textbf{Afecta} & \\ \textbf{El impacto del problema es} & \\ \textbf{Una solución sería} & \\
		\hline
	\end{tabular*}
\end{table}


\subsection{Solución Propuesta}
%Proporcionar posibles soluciones alternativas que las partes interesadas consideran viables. Enumere estas alternativas agrupadas por las partes interesadas y proporcionar una visión general de su fuerza y ​​su debilidad desde la perspectiva de que las partes interesadas.

\newpage
\section{Descripción de Usuarios y Stakeholders}
%A fin de ejecutar eficazmente los productos y servisios que satisfgan las necesidades de las partes interesadas en el proyecto, es necesario conocer quienes son los interesados y hacer que participen en el proceso de desarrollo. Un grupo importante de actores está formado por los usuarios finales del sistema. Esta sección proporciona una visión general de las partes interesadas, sus intereses y los problemas que se resuelven por la solución propuesta.

\subsection{Resumen de los Stakeholders}

\begin{table}[H]
	\centering
	\begin{tabular*}{0.75\textwidth}{c c c}
		\textbf{Nombre} & \textbf{Descripción} & \textbf{Responsabilidades} \\ \hline
		&  &  \\ \hline
		&  &  \\ \hline
		&  &  \\ \hline
	\end{tabular*}
\end{table}

\subsection{Resumen de los Usuarios}

\begin{table}[H]
	\centering
	\begin{tabular*}{0.75\textwidth}{c c c}
		\textbf{Nombre} & \textbf{Descripción} & \textbf{Stakeholder} \\ \hline
		&  &  \\ \hline
		&  &  \\ \hline
		&  &  \\ \hline
	\end{tabular*}
\end{table}

\subsection{Perfil de los Stakeholders}
%Describir cada grupo (rol), que tiene un interés en el sistema.

\begin{table}[H]
	\centering
	\begin{tabular*}{0.75\textwidth}{c | c}
		\hline
		\textbf{Descripción} & \\ \textbf{Responsabilidades} & Mencione las responsabilidades más importantes de este stakeholders con respecto al sistema. Por ejemplo,
- Este stakeholders asegura de que el sistema sea fácil de mantener;
- Este stakeholders asegura que el sistema coincide con la demanda del mercado.\\ \textbf{Criterios de éxito} & ¿Cuándo, de acuerdo con este rol podemos decir que hay éxito del proyecto? Tenga en cuenta que los criterios de éxito varían entre los roles de los stakeholders.]\\
		\hline
	\end{tabular*}
\end{table}

\subsection{Perfil de los Usuarios}

\begin{table}[H]
	\centering
	\begin{tabular*}{0.75\textwidth}{c | c}
		\hline
		\textbf{Descripción} & \\ \textbf{Responsabilidades} & \\ \textbf{Criterios de Éxito} & \\
		\hline
	\end{tabular*}
\end{table}

\subsection{Necesidades clave de los Stakeholders o Usuarios}
%Proporcionar las necesidades o problemas percibidos por los diferentes roles de los stakeholders. Esta visión general se utiliza a continuación para dar prioridad a las necesidades.
\section{Historial de Cambios}
\begin{table}[H]
	\centering
	\begin{tabular*}{\textwidth}{c c c c c}
		\textbf{Necesidades} & \textbf{Prioridad} & \textbf{Incumbre} & \textbf{Solución Actual} & \textbf{Solución Propuesta} \\ \hline
		&  &  &  & \\ \hline
		&  &  &  & \\ \hline
		&  &  &  & \\ \hline
	\end{tabular*}
\end{table}

\newpage
\section{Resuemen del Producto}

\subsection{Perspectiva del Producto}
%[Describir el producto en relación a otros productos del entorno del cliente. Si el producto es independiente, indicarlo aquí. Si el producto es un componente de un sistema más grande o se conecta a otros sistemas, mostrar cómo estos sistemas interactúan entre si. Se puede facilitar un diagrama para mostrar las relaciones.]

\subsection{Resumen de Capacidades}
%[Proporcionar una lista de características y dar prioridad a la lista. Se puede utilizar parte del modelo de casos de uso cómo descripción de este apartado. La lista de características o el modelo de casos de uso proporcionan la base para describir el alcance del proyecto.]


\subsection{Supuestos y Dependencias}
%[Proporcionar una lista de requisitos no funcionales, como la calidad, robustez, facilidad de uso, rendimiento, escalabilidad y facilidad de mantenimiento y priorizar la lista.]




\end{document}