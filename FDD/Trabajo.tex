\documentclass[11pt]{article}
\usepackage[spanish]{babel}
\usepackage{fontspec}
\usepackage{listings}
\usepackage{graphicx}
%\graphicspath{{../Imagenes/}}

\usepackage[paper=portrait, pagesize]{typearea}
\usepackage{titlepic}

%%% Tablas
\usepackage{longtable}
\usepackage{tabularx}
\usepackage{float}
\usepackage{adjustbox}
\usepackage{booktabs}
\usepackage{multirow}
\usepackage{url}
\renewcommand{\arraystretch}{1.7}

\begin{document}

\begin{titlepage}
\centering
\vspace{4.5cm}
{\scshape\LARGE Metodología Ágil:\\ Feature Driven Development\par}
\vspace{1.5cm}

%\includegraphics[width=16cm] {}

\vspace{3cm}
{\scshape\large \par}
\vspace{1cm}

{Miguel Albertí Pons\\
Sofía Almeida Bruno\\
Pedro Manuel Flores Crespo\\
María Victoria Granados Pozo\\
Lidia Martín Chica
\par}

\end{titlepage}
%\cite{SeminarioMunich}

\thispagestyle{empty}
\tableofcontents

\newpage

%%%%%%%%%%%%%%%%%%%%%%%%%%%%%%%%%%%%%%%%%%%%%%%%%%%%%%%%%
%%%% BIBLIOGRAFÍA
%%%%%%%%%%%%%%%%%%%%%%%%%%%%%%%%%%%%%%%%%%%%%%%%%%%%%%%%%
\section{Introducción}

\section{Procesos}

La metodología FDD consta de cinco fases en su proceso: Son las siguientes:

\subsection* {Construir lista de características (\textit{Build Feature List})}

Esta etapa recibe como entrada el modelo de objetos y los requerimientos (\textit{feature list informal}) obtenidos en la etapa anterior.  Estos son agrupados según el área de dominio. Cada grupo de se denomina \textit{Major List Sets}. Esta lista a su vez es dividida en otros subconjuntos según la funcionalidad. Posteriormente, cada característica o funcionalidad es priorizada y por último aquellas más complejas son divididas en otras más pequeñas. Esta etapa nos aporta como salida la \textit{Feature List} que es revisada por los usuarios para su aprobación y validación. Las tareas llevadas a cabo en esta etapa se resumen en la siguiente tabla. \\

\begin{table}[H]
\centering
\begin{tabular}{ |p{6cm}|p{6cm}|  }
	\hline
	Tarea & Responsables \\
	\hline
	Formación del
	\textit{Feature-List Team}  &    \textit{Project
		Manager} y 
	\textit{	Development
		Manager}\\
	Identificación de funcionalidades y formación del \textit{Feature Set}  & \textit{Feature-List Team}\\
	Priorización de las funcionalidades &  \textit{Feature-List Team}\\
	División de las funcionalidades complejas    & \textit{Feature-List Team}\\
	\hline
\end{tabular}
\caption{Tareas etapa de construcción de la lista de características.}
\end{table}


\section{Comparación con otros métodos}

\section{Ventajas}

\section{Roles y responsabilidades}

\newpage
\section{Bibliografía}
\nocite{*}
\bibliographystyle{plain}
\bibliography{referencias.bib}
%[type=inbook,heading=subbibliography,title={Libros}]

\end{document}
